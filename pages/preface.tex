\chapter{前言}

粒子物理学习笔记,记录了粒子物理基础知识与部分 Higss 粒子相关的学习过程。(主要是对别人知识的搬运)

\begin{itemize}
    \item 中山大学余钊焕老师的讲义
        \begin{itemize}
            \item \href{https://yzhxxzxy.github.io/teaching/2109_PartPhys_sec1_intro.pdf}{第一节\ 概述}
            \item \href{https://yzhxxzxy.github.io/teaching/2109_PartPhys_sec2_kin.pdf}{第二节\ 粒⼦运动学、衰变和散射}
            \item \href{https://yzhxxzxy.github.io/teaching/2109_PartPhys_sec3_sym.pdf}{第三节\ 对称性和守恒定律}
            \item \href{https://yzhxxzxy.github.io/teaching/2109_PartPhys_sec4_QED.pdf}{第四节\ 量子电动力学}
            \item \href{https://yzhxxzxy.github.io/teaching/2109_PartPhys_sec5_QCD.pdf}{第五节\ 量子色动力学}
            \item \href{https://yzhxxzxy.github.io/teaching/2109_PartPhys_sec6_EW.pdf}{第六节\ 电弱规范理论}
            \item \href{https://yzhxxzxy.github.io/teaching/2109_PartPhys_sec7_part.pdf}{第七节\ 典型粒子性质}
        \end{itemize}
    \item 卢昌海老师的 \href{https://www.changhai.org/articles/science/physics/origin_of_mass/}{质量的起源} 系列文章
    \item ATLAS 实验组的文章 - \href{https://inspirehep.net/literature/1124338}{Observation of a new boson at a mass of 125 GeV with the CMS experiment at the LHC}
\end{itemize}
