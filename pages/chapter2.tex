\chapter{质量起源——卢昌海}

\section{电磁质量说}

在牛顿力学中,质量是决定物体惯性与引力的基本物理量,是一个不可约的概念。在牛顿力学时期,物理学家试图把物理学的各个分支尽可能地约化为力学。很显然,在那样一个以机械观为主导的时期里,有关质量起源的研究是基本不存在的。

要想把力学约化为电磁理论,关键就要把力学中不可约的质量概念约化为电磁概念,这是物理学家们研究质量起源的第一种定量的尝试。

\section{经典电子论}

1904 年,洛伦兹发表了《任意亚光速运动系统中的电磁现象》的文章,根据电磁理论中的
包括长度收缩、局域时间在内的一系列假设,计算了具有均匀面电荷分布的运动电子的电磁动量,由此得到电子的横质量 $m_T$ 与纵质量 $m_L$

\begin{equation}
    m_T = \frac{2 e^2}{3 R c^2} \gamma, \quad m_L = \frac{2 e^2}{3 R c^2} \gamma^3
\end{equation}

\noindent 其中 $e$ 为电子的电荷,$R$ 为电子在静止参照系中的半径,$c$ 为光速,$\gamma=(1-v^2/c^2)^{-1/2}$。

\noindent 然而亚伯拉罕指出,质量除了像洛伦兹那样通过动量来定义,还应该可以通过能量来定义,比如纵质量可以定义为

\begin{equation}
    m_L = \frac{1}{v} \frac{\dif E}{\dif v}
\end{equation}

但是简单的计算指出两者并不相同,这说明洛伦兹的电子论有缺陷。

除亚伯拉罕外,另一位经典物理学大师庞加莱也注意到了洛伦兹电子论的这一问题。1904-1906 年间庞加莱亲自对洛伦兹电子论进行了研究,并定量地引进为维持电荷平衡所需的张力这种张力因而被称为庞加莱张力。在庞加莱工作的基础上,1911 年 (即在爱因斯坦与闵科夫斯基建立了狭义相对论的数学框架之后),德国物理学家劳厄证明了带有庞加莱张力的电子的能量动量具有正确的洛伦兹变换规律。

按照狭义相对论中最常用的约定,我们引进两个惯性参照系:$S$ 与 $S^\prime$,$S^\prime$ 相对于 $S$ 沿 $x$ 轴以速度 $v$ 运动。假定电子在 $S$ 系中静止,则在 $S^\prime$ 系中电子的动量为

\begin{equation}
    p^{\prime \mu} = \int T^{\prime 0\mu} (x^{\prime \xi}) \dif^3 x^\prime = L_\alpha^0 L_\beta^\mu \int T^{\alpha\beta} (x^{\xi}) \dif^3 x^\prime
\end{equation}

\noindent 其中 $T$ 为电子的总能量动量张量,$L$ 为洛伦兹变换矩阵。由于 $S$ 系中 $T^{\alpha\beta}$ 与 $t$ 无关,考虑到

\begin{equation}
    \int T^{\alpha\beta} (x^\xi) \dif^3 x^\prime = \int T^{\alpha\beta} (\gamma x^\prime,y^\prime,z^\prime) \dif^3 x^\prime = \gamma^{-1} \int T^{\alpha\beta} (x^\xi) \dif^3 x
\end{equation}

由此得到电子的能量与动量分别为

\begin{align}
    E &= p^{\prime 0} = \gamma m + \gamma^{-1} L_i^0 L_j^0 \int T^{ij} (x^\xi) \dif^3 x \\
    P &= p^{\prime 1} = \gamma v m + \gamma^{-1} L_i^0 L_j^1 \int T^{ij} (x^\xi) \dif^3 x
\end{align}

这里 $i,j$ 为空间指标 $1,2,3$,$m=\int T^{00} (x^\xi) \dif^3 x$,为了简化结果,我们取 $c=1$。

那么庞加莱张力为什么能够避免洛伦兹电子论的问题呢?关键在于引进庞加莱张力后电子才成为一个满足力密度 $f^\mu = \partial_\nu T^{\mu\nu}=0$ 的孤立平衡体系。在电子静止系 $S$ 中 $T^{\mu\nu}$ 不含时间,因此 $\partial_j T^{ij}=0$。由此可以得到一个很有用的关系式 $\partial_k (T^{ik} x^j)=T^{ij}$。对此式做体积分,注意到左边的积分为零,可得

\begin{equation}
    \int T^{ij} (x^\xi) \dif^3 x = 0
\end{equation}

这个结果称为劳厄定理,它表明我们上面给出的电子能量动量表达式中的第二项为零。因此庞加莱张力的引进非常漂亮地保证了电子能量动量的协变性。

典电子论似乎达到了一个颇为优美的境界,但事实上失败了,因为那个“非常漂亮地”保证了电子能量动量协变性的庞加莱张力必须是非电磁起源的,就这样,试图把质量约化为纯电磁概念的努力由于必须引进非电磁起源的庞加莱张力而化为了泡影。

\section{量子电动力学}

量子理论对经典物理学的冲击是全方位的。就经典电子论中有关电子结构的部分而言,对这种冲击最简单的描述来自 \emph{不确定性原理}。如我们在上一节中所述,经典电子论给
出的电子质量 (除去一个与电荷分布有关的数量级为 $1$ 的因子) 约为 $e^2/R c^2$,由此可以很容易地估算出 $R \approx \qty{e-15}{m}$。这一数值被称为电子的经典半径。但是从不确定性原理的角度看,对电子空间定位的精度只能达到电子的康普顿波长 $\hbar/mc \approx R/\alpha \approx \qty{e-12}{m}$ 的量级 (其中 $\alpha\approx 1/137$ 为精细结构常数),把电子视为经典电荷分布的做法只有在空间尺度远大于这一量级的情形下才适用。由于电子的经典半径远远小于这一尺度,这表明经典电子论并不适于描述电子的结构。建立在经典电子论基础上的电子质量计算也因此失去理论基础。

虽然如此,但是那种计算所体现的自相互作用对电子质量产生贡献的思想却是合理的,并在量子
理论中得到保留,这种贡献称为 \emph{电子自能}。

在现代量子场论中,相互作用对电子自能的主要贡献来自由量子电动力学所描述的电磁自能,而电磁自能中最简单的贡献则来自单圈图。幸运的是,由于量子电动力学的耦合常数在所有实验所及的能区都很小,因此这个最简单的单圈图的贡献在整个电子自能中占主要部分。

这一单圈图的计算在任何一本量子场论教材中都有详细介绍,其结果为 $\delta m \approx \alpha m \ln(\Lambda/m)$,其中 $m$ 为出现在量子电动力学拉氏量中的电子质量参数,称为裸质量,$\Lambda$ 为虚光子动量的截断。如果我们把量子电动力学的适用范围无限外推,允许虚光子具有任意大的动量,则 $\delta m$ 将趋于无穷,这便是自 20 世纪 30 年代起困扰物理学界几十年之久的量子场论发散困难的一个例子。

基本粒子在量子场论中是以点粒子的形式出现的,虽然这并不意味着它们不具有唯象意义上的等效结构,但所有那些结构都是作为理论的结果而不是如经典电子论中那样作为额外假设而出现的,这是除与狭义相对论及量子理论同时兼容、与实验高度相符之外,建立在点粒子模型基础上的量子场论又一个明显优于经典电子论的地方。

至于由此产生的发散困难,在 20 世纪 70 年代之后得到了较为系统的解决。这一解决方法被称为重整化方法。不过,尽管重整化方法无论在数学计算还是物理意义的理解,都已相当成熟,但发散结果的存在基本上消除了传统量子场论成为所谓终极理论的可能性。

\section{质量电磁起源的破灭}

一方面,从 $\delta m\approx \alpha m \ln(\Lambda/m)$ 中的 $\alpha \ln(\Lambda/m)$ 部分可以看到,由于 $\alpha\approx 1/137$ 是一个很小的数目,而 $\ln(\Lambda/m)$ 又是一个增长极其缓慢的函数,因此对于任何普朗克能标以下的截断,由电磁自能产生的质量修正与所谓的裸质量 $m$ 相比都只占一个很小的比例。

另一方面,即使我们一厢情愿地把量子电动力学的适用范围延伸到比普朗克能标还高得多的能区,以致 $\delta m$ 变得很大,把质量完全约化为电磁概念的梦想也依然无法实现。因为电子的电磁自能还有一个要命的特点,即 $\delta m \propto m$。这表明,无论把截断取得多大,如果裸质量为零,电子的电磁自能也将为零。而裸质量是量子电动力学拉氏量中的参数,在量子电动力学范围内无法约化。

有的读者可能会问:电磁自能既然是由电磁相互作用引起的,理应只与电荷有关,为什么却会正比于裸质量呢?这其中的奥妙在于对称性。量子电动力学的拉氏量

\begin{equation}
    \mathcal{L} = -\frac{1}{4} F^{\mu\nu} F_{\mu\nu} + \overline{\psi}(i\gamma^\mu\partial_\mu - m) \psi - e \overline{\psi} \gamma^\mu A_\mu \psi
\end{equation}

在 $m=0$ 的时候具有一种额外的对称性,即在 $\psi \to e^{i\alpha\gamma^5} \psi$ 下不变。这种对称性称为手征对称性,它表明在 $m=0$ 的情形下电子的左右手征态

\begin{equation}
    \psi_L = (1-\gamma^5) \psi/2, \quad \psi_L = (1+\gamma^5) \psi/2
\end{equation}

\noindent 不会互相耦合。另一方面,电子的质量项

\begin{equation}
    m \overline{\psi} \psi = m \overline{\psi}_L \psi_R + m \overline{\psi}_R \psi_L
\end{equation}

却是一个电子左右手征态相互耦合,从而破坏手征对称性的项。这样的项在电子的裸质量不存在 (从而量子电动力学的拉氏量具有手征对称性) 的情况下将被手征对称性所禁止,不可能出现在任何微扰修正中。

试图把质量完全归因于电磁相互作用的想法在量子场论中彻底破灭了。电磁质量即使在像电子这样质量最小 (从某种意义上讲也最为纯粹) 的带电粒子的质量中也只占一个不大的比例,在其他粒子 (尤其是那些不带电荷的基本粒子) 中就更甭提了。很显然,质量的主要来源必须到别处去寻找。

\section{对称性自发破缺}

在质量的电磁起源破灭后,质量起源问题沉寂了很长一段时间。物理学再次回到质量起源问题是在 20 世纪 60 年代。物理学家在对基本粒子的研究中已经发现了许多对称性。对称性在物理学中一直有着重要的地位,不仅由于其优美的形式与物理学家们对自然规律的美学追求十分吻合,更重要的是因为它不仅中看,而且中用,有一种穿透复杂性的力量。

即使在对体系动力学行为还没有透彻理解的情况下,对称性也往往具有令人瞩目的预言能力。这最后一点在 20 世纪五六十年代的粒子物理研究中具有极大的吸引力,因为当时人们对基本粒子相互作用的动力学机制知之甚少,而且对在很大程度上为研究基本粒子相互作用而发展起来的量子场论产生了很深的怀疑。在这种情况下,许多物理学家对对称性寄予了厚望,希望通过它们来窥视大自然在这一层次上的奥秘。

但不幸的是,当时所发现的许多对称性却被证明只在近似情况下成立,比如同位旋对称性与宇称对称性。如何理解这种近似对称性呢?当时有一种猜测,认为近似对称性是 (严格) 对称性自发破缺的产物。

所谓对称性自发破缺,指的是这样一种情形:

\begin{quote}
    一个物理体系的拉氏量具有某种对称性,而基态却不具有该对称性。换句话说,体系的基态破缺了运动方程所具有的对称性。
\end{quote}

这种对称性自发破缺的概念最早出现在凝聚态物理中,20 世纪 60 年代被南部阳一郎等引入量子场论。在量子场论中,体系的基态是真空态,因此对称性自发破缺表现为体系拉氏量所具有的对称性被真空态所破缺。有的读者可能会问:一个物理体系的真空态是:由拉氏量所确定的,为什么会不具有拉氏量所具有的,的对称性呢?这其中的奥秘在于,许多物理体系具有 \emph{简并} 的真空态,如果我们把所有这些简并的真空态,视为一个集合,它的确与拉氏量具有同样的对称性。

但物理体系的实际真空态只是该集合中的一个态,这个态往往不具有整个集合所具有的对称性,这就造成了对称性的破缺,也就是我们所说的 \emph{对称性自发破缺}。

把近似对称性归因于对称性自发破缺的想法在 1961 年曾遭到致命打击。那一年由戈德斯通
提出并在稍后与萨拉姆及温伯格一起证明了这样一个命题 (被称为戈德斯通定理):

\begin{quote}
    每一个自发破缺的整体连续对称性都必然伴随一个 \emph{无质量标量粒子} (被称为 \emph{戈德斯通粒子} 或 \emph{南部-戈德斯通粒子})。
\end{quote}

为什么会有这样的结果呢?我们来简单地证明一下:假定拉氏量中的势函数为

\begin{equation}
    V(\varphi_a) \quad (a=1, \cdots, N)
\end{equation}

\noindent 其中 $\varphi_a$ 为标量场 (可以是基本的、也可以是复合的)。

显然这一体系的真空态满足 $\partial V/\partial \varphi_a=0$,而标量粒子的质量 (平方) 由 $\partial^2 V/\partial \varphi_a \partial \varphi_b$ 在真空态上的本征值给出。

现在考虑对真空态 $\varphi_a$ 作一个无穷小连续对称变换

\begin{equation}
    \varphi_a \to \varphi_a + \varepsilon \Delta_a(\varphi)
\end{equation}

\noindent 其中 $\varepsilon$ 为无穷小参数。

由于 $V(\varphi_a)$ 在这一变换下不变,因此有:

\begin{equation}
    \Delta_a(\varphi) (\partial V/\partial \varphi_a)=0
\end{equation}

\noindent 将这一表达式对 $\varphi_b$ 作一次导数,并注意到真空所满足的条件,可得:

\begin{equation}
    \Delta_a(\varphi) \partial^2 V/\partial \varphi_a \partial \varphi_b=0
\end{equation}

\noindent 由上式可以看到,每一个 $\Delta_a(\varphi) \ne 0$ 的连续对称变换都对应于 $\partial^2 V/\partial \varphi_a \partial \varphi_b$ 的一个本征值为零的本征态,从而也就对应于一个无质量标量粒子。而 $\Delta_a(\varphi) \ne 0$ 的连续对称变换所对应的正是那些不能使真空态不变 (从而被真空态所破缺,即自发破缺) 的连续对称性。这就证明了每一个自发破缺的整体连续对称性都必然伴随一个无质量标量粒子 (即戈德斯通粒子),这正是戈德斯通定理。

戈德斯通定理也可以从几何上来理解。$V=V(\phi_a), a=1, 2, \cdots, N$ 可以看成是一个 $N$ 维曲面,真空态对应于该曲面的一个极小值点,而该点处每一个独立的平坦方向 (即二阶导数为零的方向) 对应于一个无质量标量粒子。另一方面,每一个这种独立的平坦方向对应于一个可以使真空态移到邻近点的连续对称变换。这种连续对称变换所表示的正是被真空态所破缺的对称性。这就表明无质量标量粒子与这种自发破缺的对称性一一对应。

由于自发破缺的整体连续对称性的数目等于这些对称性的生成元数目,因此戈德斯通定理表明戈德斯通粒子的数目等于自发破缺的整体连续对称性的生成元数目。举个例子来说,$SU(2)$ 对称性具有 $3$ 个生成元,若完全破缺,就会产生 $3$ 个戈德斯通粒子;若破缺为 $U(1)$,则只产生 $2$ 个戈德斯通粒子 (因为有一个生成元未破缺)。进一步的分析还表明,戈德斯通粒子与那些自发破缺的整体连续对称性所对应的荷具有相同的宇称及内禀量子数。

为什么会对把近似对称性归因于对称性自发破缺的想法造成致命打击呢?原因很简单,那就是近似对称性中有一些正是整体连续对称性 (比如同位旋对称性),如果它们果真来源于对称性自发破缺的话,那就应该存在相应的无质量标量粒子。但我们从未在实验上观测到任何这样的粒子。因此对称性自发破缺的想法在粒子物理学中由于牵涉到无质量粒子而陷入困境。

\section{从希格斯机制到电弱统一理论}

1954 年由杨振宁和米尔斯提出,现在被称为 杨-米尔斯理论 的定域非阿贝尔规范理论。这种理论是对量子电动力学所具有的定域 $U(1)$ 规范对称性的推广,最初是想用来描述同位旋对称性。

但它立刻就遇到很大的困难,那便是这种理论所具有的定域规范不变性会无可避免地导致无质量的矢量粒子 (被称为规范粒子,类似于量子电动力学中的光子),而在现实中,除光子外,我们从未在实验上观测到任何这样的无质量矢量粒子。

杨-米尔斯理论与对称性自发破缺这两个出色的想法先后搁浅了,追根溯源,都是无质量粒子惹的祸。

对称性自发破缺的问题出在哪里呢?出在整体连续对称性上;而杨-米尔斯理论的问题又出在哪里呢?出在定域规范对称性 (那是一种特殊的定域连续对称性) 上。如果我们把这两者放在一起,让对称性自发破缺干掉那些产生无质量矢量粒子的定域规范对称性,杨-米尔斯理论不就可以摆脱困境了吗?更妙的是,由于杨-米尔斯理论中的对称性不是整体而是定域的,戈德斯通定理将不适用于这种对称性的自发破缺,这样一来说不定那些可恶的戈德斯通粒子也会消失,那岂不是两全其美?世界上会有这么好的事吗?还真的有。

% 乐

凝聚态大佬安德森意 在 1963 年猜测道:

\begin{quote}
    “戈德斯通的零质量困难并不是一个严重的困难,因为我们很可能可以用一个相应的杨-米尔斯零质量问题来消去它”。
\end{quote}

用技术性的语言来说,希格斯机制中对应于戈德斯通粒子的那些自由度可以被定域规范变换所消去 (必须注意的是:“定域”二字在这里至关重要,整体的连续变换是不具有这种能力的)。从 \emph{规范理论} 的角度讲,这相当于选取了一种被称为幺正规范的特殊规范。这种特殊规范的选取造成定域规范对称性的破缺,从而使原本受定域规范对称性所限必须无质量的规范粒子可以获得质量。人们有时把这种机制形象地描述为:

\begin{quote}
    规范粒子通过 \emph{吃掉} 戈德斯通粒子而获得质量。
\end{quote}

希格斯机制不仅一举救活了粒子物理学中对称性自发破缺与杨-米尔斯理论这两个极为出色的想法,而且在救助过程中为我们提供了一种产生质量的新方法,即通过规范对称性的自发破缺,从不带质量项的拉氏量中产生出质量来。我们在上面已经提到规范粒子可以由此获得质量。不过规范粒子在宇宙可见物质的质量中所占的比例极小,我们更关心的是在可见物质质量中占主要比例的那些粒子——费米子。在后来建立起来的粒子物理标准模型中,费米子也是通过规范对称性的自发破缺,或者更确切地说,通过电弱统一理论中的规范对称性自发破缺获得质量的,这其中希格斯机制起到了重要作用。具体地讲,在标准模型中费米场与标量场 (也称为希格斯场) 之间存在所谓的汤川耦合。

遗憾的是,这一回答却是一个不尽人意的回答。为什么这么说呢?因为这一回答与其说是在回答问题,不如说是在转嫁问题,它只是把我们想要理解的基本粒子的质量值转嫁给了希格斯场的真空期待值、规范耦合常数以及汤川耦合常数。这其中希格斯场的真空期待值及规范耦合常数与基本粒子 (主要是费米子) 的种类无关,可以算是普适的,因此将质量向这些参数约化不失为是一种有效的概念约化。但汤川耦合常数则不然,它对于每一种费米子都有一个独立的数值。由于这些参数的存在,标准模型的拉氏量虽然不显含质量参数,但它所包含的与质量直接有关的自由参数数目却一点也不比原先需要解释的质量参数数目来得少 (事实上还略多一点)。

\section{量子色动力学}

与戈德斯通、希格斯等人在对称性自发破缺方面的研究几乎同时,物理学家们在研究强相互作用上也取得了重大进展。经过十余年的努力,这些工作最终奠定了被称为量子色动力学的强相互作用理论。量子色动力学是一个以 $SU(3)$ 为规范群的杨-米尔斯理论,它所描述的费米子被称为夸克,是有质量粒子 (但其质量在标准模型之内是不可约化的);传递相互作用的载力子则是无质量的,被称为 \emph{胶子}。

在谈论质量起源问题的时侯,人们往往把注意力放在包含希格斯机制的电弱统一理论上,因为希格斯机制在登场伊始就打出了质量产生机制的响亮广告。但事实上我们将会看到,看似与质量起源问题无关的量子色动力学对这一问题有着非常独特而精彩的回答,从某种意义上讲,这一回答才是标准模型范围内的最佳回答。

量子色动力学的拉氏量:

\begin{equation}
    L = -\frac{1}{2} Tr(G^{\mu\nu} G_{\mu\nu}) + \sum_q \overline{q}(i\gamma^\mu D_\mu - m_q)q
\end{equation}

\noindent 其中 $q$ 为夸克场,$G^{\mu\nu}=\partial_\mu A_\nu - \partial_\nu A_\mu - i g [A_\mu, A_\nu]$ 为规范场强,$D_\mu=\partial_\mu-i g A_\mu$ 为协变导数,$A_\mu$ 为规范势,$m_q$ 为夸克 $q$ 的质量,$g$ 为耦合常数。自然界已知的夸克种类 (也称为“味”) 共有六种。

在接下来的几节中,我们就来看一下量子色动力学对强子质量的描述,以及这种描述在何种意义上可以被视为是对质量起源问题的回答。

\section{同位旋与手征对称性}

我们知道,可见物质的质量主要来自于质子和中子,其中质子由两个 $u$ 夸克及一个 $d$ 夸克组成,而中子由一个 $u$ 夸克及两个 $d$ 夸克组成。在下面的叙述中,我们将只考虑这两种夸克。由于这两种夸克的质量远小于任何强子的质量,作为近似,我们先忽略它们的质量。这时的量子色动力学拉氏量为:

\begin{equation}
    L = -\frac{1}{2} \mathrm{Tr} G^{\mu\nu} G_{\mu\nu} + i \overline{u} \gamma^\mu D_\mu u + i \overline{d} \gamma_\mu D_\mu d
\end{equation}

\noindent 显然,这一拉氏量在以下两个整体 $SU(2)$ 变换

\begin{equation}
    \psi \to e^{-i t^a \theta^a} \psi, \quad \psi \to e^{-i \gamma^5 t^a \theta^a} \psi
\end{equation}

\noindent 下是不变的。其中 $\psi=(u, d)^T$,$t^a$ 是 $SU(2)$ 的生成元。这两个存在于 $u$ 和 $d$ 夸克间的对称性分别被称为同位旋对称性与手征对称性,记为 $SU(2)_V$ 与 $SU(2)_A$ 。其中,同位旋对称性 $SU(2)_V$ 只要夸克质量彼此相等 (不一定要为零) 就存在,而手征对称性 $SU(2)_A$ 只有在夸克质量全都为零时才具有 (这一情形因此而被称为手征极限) 。除此之外,这一拉氏量还存在一个显而易见的整体 $U_V(1)$ 对称性,它对应于重子数守恒,与夸克是否有质量,以及质量是否彼此相等都无关。综合起来,上述拉氏量具有整体 $SU(2)_V \times SU(2)_A \times U(1)_V$ 对称性。在这些对称性中,同位旋对称性 $SU(2)_V$ 与手征对称性 $SU(2)A$ 所对应的守恒流分别为

\begin{equation}
    V^{\mu a} = \overline{\psi} \gamma^\mu t^a \psi, \quad A^{\mu a} = \overline{\psi} \gamma^\mu \gamma^5 t^a \psi
\end{equation}

\noindent 显然在宇称变换下, $V^{\mu a}$ 是矢量、$A^{\mu a}$ 是轴矢量。他们对应的荷 $(Q_V)^a=\int V^{0a} \dif^3 x$ 与 $(Q_A)^a=\int A^{0a} \dif^3 x$ 分别为标量及赝标量。

如果同位旋与手征对称性都是严格的对称性,那么 $(Q_V)^a$ 将生成强子谱中自 20 世纪 60 年代起逐步引导人们发现量子色动力学的同位旋对称性;而 $(Q_A)^a$ 则将生成所谓的手征对称性,它要求每一个强子都伴随有自旋、重子数及质量与之相同,而宇称却相反的粒子。这样的对称性在强子谱中从未被发现过。

对此,最容易想到的解释是:由于 $u$ 和 $d$ 夸克实际上 \emph{并不是无质量的},因此手征对称性原本就不可能严格成立。事实上,不仅手征对称性不可能严格成立,由于 $u$ 夸克和 $d$ 夸克的质量彼此不同,连同位旋对称性也不可能严格成立。但是,考虑到 $u$ 夸克和 $d$ 夸克的质量相对于强子质量是如此之小,相应的对称性在强子谱中似乎起码应该近似地存在。对于同位旋对称性来说,情况的确如此 (否则就不会有早年那些强子分类模型了) 。但手征对称性却哪怕在近似意义上也根本不存在。举个例子来说,手征对称性要求介子三重态 $\rho(770)$ 与 $a_1(1260)$ 互为对称伙伴,但实际上这两者的质量分别为 \qty{775}{MeV} 和 \qty{1230}{MeV},相差悬殊 (作为对比,同位旋伙伴的质量差通常都在几个 \unit{MeV} 以下) , 连近似的对称性也不存在。

\section{手征对称性自发破缺}

手征 $SU(2)$ 是量子色动力学拉氏量中的 (近似) 对称性,却在现实世界中完全找不到对应。

对称性自发破缺在电弱统一理论中用得好好的,为什么在量子色动力学中却变得“极其困难”了呢? 这是因为在电弱统一理论中对称性自发破缺是由人为引进的希格斯场产生的,我们有一定的自由度来选择对称性破缺的方式。但量子色动力学并不包含这种人为引进的希格斯场,因此在量子色动力学中,整体 $SU(2)_V \times SU(2)_A \times U(1)_V$ 对称性是否自发破缺?如果破缺,是否恰好是手征部分破缺,即破缺到 $SU(2)_V \times U(1)_V$?都只能由理论本身来决定,而不是我们可以擅自假设的,正是这一特点使问题变得困难。更麻烦的是,手征对称性的破缺 (如果存在的话) 出现在量子色动力学的强相互作用区,即低能区。对于理论研究来说,无疑是雪上加霜。

那么量子色动力学究竟能否实现从 $SU(2)_V \times SU(2)_A \times U(1)_V$ 到 $SU(2)_V \times U(1)_V$ 的对称性自发破缺呢?目前在理论上还是待解之谜。

\begin{enumerate}
    \item 1979 年,特霍夫特通过对规范理论中的反常进行分析,得到一个结果:即如果所考虑的整体对称性是 $SU(3)_V \times S U (3) A \times U(1)_V$ ,那它就必须自发破缺。可惜,一来量子色动力学中的 $SU(3)$ 对称性远比 $SU(2)$ 对称性粗糙,二来这一结果也无法告诉我们具体哪一部分对称性会自发破缺。
    \item 1980 年,柯曼与威腾提出在某些合理的物理条件下,当色的数目 $N_c$ 趋于无穷时,手征对称性必须自发破缺。这一结果虽然抓准了手征对称性,但可惜量子色动力学中的 $N_c$ 不仅不是无穷,而且还很小 $(N_c=3)$ 。
    \item 1984 年,瓦法与威腾证明了未被非零夸克质量项所破缺的同位旋对称性不会自发破缺。可惜这一证明虽然表明特定的同位旋对称性不会自发破缺,却未对手征对称性是否一定会自发破缺提供说明。
\end{enumerate}

虽然上述理论研究并未能证明 $SU(2)_V \times SU(2)_A \times U(1)_V$ 必定会破缺到 $SU(2)_V \times U(1)_V$,但它们都与这一对称性破缺方式相容。从大量间接证据来看,它的证明应该只是时间问题。物理学家更感兴趣的是:如果手征对称性自发破缺,我们可以从中得到什么推论。

\section{赝戈德斯通粒子的质量}

对称性自发破缺最重要的推论之一是存在无质量的标量粒子,即戈德斯通粒子,它们与破缺对称性所对应的荷具有相同的宇称及内禀量子数。对于手征对称性来说,荷是 $(Q_A)^a$,它在时空中是赝标量,在内禀空间中则是矢量。因此相应的戈德斯通粒子的宇称为负,同位旋则为 $1$ 。自然界满足上述特征的强子中质量最轻的是 $\pi$ 介子 $(\pi^-, \pi^0, \pi^+)$ 。如果手征对称性是自发破缺的,$\pi$ 介子就应该是这一破缺所对应的戈德斯通粒子。但是,戈德斯通粒子是无质量的,$\pi$ 介子却是有质量的,这一矛盾该如何解决呢?

在理想的对称性自发破缺情形下,体系的实际真空态可以是一系列简并真空态中的任何一个。但是,量子色动力学中的手征对称性破缺却不是理想情形,因为量子色动力学的拉氏量含有手征对称性的明显破缺项,即夸克质量项。由于这种明显破缺项的存在,实际真空态的选取就不再是任意的了,明显破缺项的存在将会对实际真空态起到一个选择作用。这就好比一根立在桌上的筷子,如果桌子是严格水平的,它向任何一个方向倒下都是同等可能的,但如果桌子是倾斜的,它就会往倾斜度最大 (梯度最大) 的方向倒。用数学语言来说,如果 $V_1(\phi_a) (a=1, \cdots, N)$ 表示对称性的明显破缺项,那么它所选出的真空态将满足 $\Delta_\alpha(\phi) (\partial V_1/\partial \phi_\alpha)=0$,这一条件被称为 \emph{真空取向条件}。另一方面,明显破缺项的存在也破坏了戈德斯通定理成立的条件,由此导致的结果是戈德斯通粒子有可能具有非零质量,这样的粒子被称为 \emph{赝戈德斯通粒子}。

真空取向条件是确定赝戈德斯通粒子质量的重要条件。赝戈德斯通粒子的出现消除了 $\pi$ 介子的非零质量与戈德斯通粒子的零质量之间的定性矛盾。但在定量上 $\pi$ 介子与赝戈德斯通粒子的质量是否吻合呢?

对于量子色动力学中的手征对称性来说,对称性的明显破缺项为质量项,它可以改写成

\begin{equation}
    V_1 = \frac{1}{2} (m_u+m_d) \overline{\psi} \psi + \frac{1}{2} (m_u-m_d) (\overline{u}u-\overline{d}d)
\end{equation}

\noindent 其中 $\overline{\psi}\psi=\overline{u}u+\overline{d}d$。上式第一项只破坏手征对称性,第二项则破坏同位旋对称性。在此基础上,考虑到不存在同位旋对称性自发破缺这一限制,可以得到赝戈德斯通粒子的质量为 \footnote{这一结果也可以从手征微扰理论得到。}

\begin{equation}
    M_{\pi}^2 = \frac{m_u+m_d}{2 F_{\pi}^2} \Braket{ 0 | \overline{\psi}\psi | 0}
\end{equation}

\noindent 其中 $F_\pi$ 是一个量纲为能量的常数,由 $\Braket{ 0 | A^{\mu a}(x) | \pi^b(p) } = i p^\mu F_\pi \delta^{ab} e^{-ipx}$ 定义。$F_\pi$ 被称为 $\pi$ 衰变常数,可以由 $\pi$ 介子的衰变来确定,原则上也可以从理论上计算出,其数值约为 \qty{9214}{MeV}。$\Braket{ 0 | \overline{\psi}\psi | 0 }$ 是一个量纲为能量 $3$ 次方的参数,被称为 \emph{手征凝聚},目前人们对它的计算还较粗略,结果大致为 $\Braket{ 0 | \overline{\psi}\psi | 0 } \approx (\qty{270}{MeV})^3 n_f$,其中 $n_f$ 为参与凝聚的夸克种类,对于我们所考虑的情形 $n_f=2$ ($u$ 和 $d$ 夸克) 。$m_u+m_d$ 通常取为 \qtyrange{8}{9}{MeV}。由此可得 $M_\pi\approx\qty{140}{MeV}$。这几乎正好就是 $\pi$ 介子的质量。

可惜的是,这种回答与我们以前介绍的电磁自能具有相同的缺陷,那就是它正比于在理论中无法约化的外来参数———夸克质量。一旦外来参数不存在,这一回答就会失效。因此量子色动力学对 $\pi$ 介子及其他赝戈德斯通粒子质量的计算虽然很漂亮,但从回答本原问题的角度看却仍不能令人满意。

\section{一个 93 分的答案}

计算核子或其他重子的质量是一个相当困难的低能量子色动力学的问题,通常的做法是利用巨型计算机进行格点量子色动力学计算。但是由于技术上的限制,迄今为止,在这类格点量子色动力学计算中采用的 $u$ 夸克和 $d$ 夸克质量都在它们实际质量的 $5$ 倍以上,由此得到的核子质量通常也要比实际值高出 $30\%$ 以上。

另一方面,与格点量子色动力学计算中夸克质量的“不可承受之重”截然相反,在前面提到的手征微扰理论中,夸克的质量却是越轻越好,甚至最好是零。显然,如果我们能在这两种极端之间作某种调和,借助手征微扰理论对格点量子色动力学计算进行适当外推,就有可能得到更接近现实世界的结果。这正是物理学家在计算核子质量时采用的手段。这种借助手征微扰理论对格点量子色动力学计算进行外推的方法被称为手征外推。利用手征外推得到的核子质量 $m_N=m_0-4 c_1 m_\pi^2 + O(m_pi^3)$。其中 $m_0 \approx \qty{880}{MeV}, c_1 \approx-\qty{1}{GeV^{-1}}$,$m^2_\pi$ 是 $\pi$ 介子的质量平方,如上节所述,正比于夸克质量。

将有关数据代入这一公式,可得 $m_N\approx\qty{954}{MeV}$,它与实际的核子质量 (质子约为 \qty{938}{MeV}、中子约为 \qty{940}{MeV}) 相当接近。

从上面所引的核子质量公式中可以看到,上述核子质量有一个不同于赝戈德斯通粒子质量的至关重要的特点,那就是它在手征极限 (即夸克质量为零) 时不为零,而等于 $m_0 \approx\qty{880}{MeV}$。这个数值约为核子质量的 $93\%$,它完全由量子色动力学所描述的相互作用所确定。这表明,即便不引进任何外来的夸克质量,量子色动力学仍能给出核子质量的绝大部分。由于宇宙中可见物质的质量主要来自核子质量,因此宇宙中可见物质质量的绝大部分都可以在不引进夸克质量的情况下,由纯粹的量子色动力学加以说明。从这个意义上讲,量子色动力学为质量起源问题提供了一个独特而精彩的回答。不过,由于它只能给出核子质量的 $93\%$ ,因此我们粗略地给它打 $93$ 分。在标准模型的范围内,这是迄今所知的最佳回答。
